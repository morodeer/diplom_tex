%\section{Создание конечно-элементной модели проектируемого самолета}

В ходе работы были исследованы вопросы построения проектировочной модели БПЛА с крылом большого удлинения и несущим фюзеляжем. При помощи программного комплекса ``Conver'' (см. раздел \ref{sec:Conver}), исходя из концептуальной модели, предложенной конструкторами, была создана МКЭ-модель проектируемого БПЛА с исключенной верхней частью воздухозаборника, не несущей в себе силовых элементов. 

\begin{figure}[ht]
\centering
\includegraphics[width=0.8\textwidth]{BPLAfullModel}
\caption{МКЭ-модель проектируемого БПЛА без верхней части}
\label{fig:BPLAfullModel}
\end{figure}

\subsection{Подбор оптимальной дискретности модели}

В целях обеспечения точности расчета было проведено исследование зависимости напряженно-деформированного состояния самолета от максимального характерного размера конечных элементов, используемых в модели. 

С помощью программного комплекса ``Conver'' было построено 7 моделей самолета по одинаковой схеме с использованием различных размеров конечного элемента. Путем расчета моделей были определены средние величины напряжений для панелей и стенок в наиболее напряженных отсеках самолета (обозначены белым на  Рис.\ref{fig:WingRootPlain})

\begin{figure}[ht]
\centering
\includegraphics[width=0.5\textwidth]{RootOfWingWithSelectedPartsBW}
\caption{Схематичное изображение вида сверху в месте стыка правого крыла и фюзеляжа}
\label{fig:WingRootPlain}
\end{figure}




Исходя из полученных данных, была получена зависимость величины средних напряжений в панелях и стенках этих отсеков от выбора размера конечного элемента (Рис.\ref{fig:stressToDiscreteness})

\begin{figure}[H]
\centering
%\includegraphics[width=0.8\textwidth]{StressToDiscretenessPlot}
<?xml version="1.0" encoding="utf-8"  standalone="no"?>
<!DOCTYPE svg PUBLIC "-//W3C//DTD SVG 1.1//EN" 
 "http://www.w3.org/Graphics/SVG/1.1/DTD/svg11.dtd">
<svg 
 width="800" height="400" 
 viewBox="0 0 800 400"
 xmlns="http://www.w3.org/2000/svg"
 xmlns:xlink="http://www.w3.org/1999/xlink"
>

<title>Gnuplot</title>
<desc>Produced by GNUPLOT 4.6 patchlevel 5 </desc>

<g id="gnuplot_canvas">

<rect x="0" y="0" width="800" height="400" fill="none"/>
<defs>

	<circle id='gpDot' r='0.5' stroke-width='0.5'/>
	<path id='gpPt0' stroke-width='0.222' stroke='currentColor' d='M-1,0 h2 M0,-1 v2'/>
	<path id='gpPt1' stroke-width='0.222' stroke='currentColor' d='M-1,-1 L1,1 M1,-1 L-1,1'/>
	<path id='gpPt2' stroke-width='0.222' stroke='currentColor' d='M-1,0 L1,0 M0,-1 L0,1 M-1,-1 L1,1 M-1,1 L1,-1'/>
	<rect id='gpPt3' stroke-width='0.222' stroke='currentColor' x='-1' y='-1' width='2' height='2'/>
	<rect id='gpPt4' stroke-width='0.222' stroke='currentColor' fill='currentColor' x='-1' y='-1' width='2' height='2'/>
	<circle id='gpPt5' stroke-width='0.222' stroke='currentColor' cx='0' cy='0' r='1'/>
	<use xlink:href='#gpPt5' id='gpPt6' fill='currentColor' stroke='none'/>
	<path id='gpPt7' stroke-width='0.222' stroke='currentColor' d='M0,-1.33 L-1.33,0.67 L1.33,0.67 z'/>
	<use xlink:href='#gpPt7' id='gpPt8' fill='currentColor' stroke='none'/>
	<use xlink:href='#gpPt7' id='gpPt9' stroke='currentColor' transform='rotate(180)'/>
	<use xlink:href='#gpPt9' id='gpPt10' fill='currentColor' stroke='none'/>
	<use xlink:href='#gpPt3' id='gpPt11' stroke='currentColor' transform='rotate(45)'/>
	<use xlink:href='#gpPt11' id='gpPt12' fill='currentColor' stroke='none'/>
</defs>
<g style="fill:none; color:white; stroke:currentColor; stroke-width:1.00; stroke-linecap:butt; stroke-linejoin:miter">
</g>
<g style="fill:none; color:black; stroke:currentColor; stroke-width:1.00; stroke-linecap:butt; stroke-linejoin:miter">
	<path stroke='black'  d='M63.6,325.3 L72.6,325.3 M775.0,325.3 L766.0,325.3  '/>	<g transform="translate(55.3,329.8)" style="stroke:none; fill:black; font-family:Arial; font-size:12.00pt; text-anchor:end">
		<text> 16</text>
	</g>
	<path stroke='black'  d='M63.6,291.0 L72.6,291.0 M775.0,291.0 L766.0,291.0  '/>	<g transform="translate(55.3,295.5)" style="stroke:none; fill:black; font-family:Arial; font-size:12.00pt; text-anchor:end">
		<text> 18</text>
	</g>
	<path stroke='black'  d='M63.6,256.7 L72.6,256.7 M775.0,256.7 L766.0,256.7  '/>	<g transform="translate(55.3,261.2)" style="stroke:none; fill:black; font-family:Arial; font-size:12.00pt; text-anchor:end">
		<text> 20</text>
	</g>
	<path stroke='black'  d='M63.6,222.4 L72.6,222.4 M775.0,222.4 L766.0,222.4  '/>	<g transform="translate(55.3,226.9)" style="stroke:none; fill:black; font-family:Arial; font-size:12.00pt; text-anchor:end">
		<text> 22</text>
	</g>
	<path stroke='black'  d='M63.6,188.1 L72.6,188.1 M775.0,188.1 L766.0,188.1  '/>	<g transform="translate(55.3,192.6)" style="stroke:none; fill:black; font-family:Arial; font-size:12.00pt; text-anchor:end">
		<text> 24</text>
	</g>
	<path stroke='black'  d='M63.6,153.8 L72.6,153.8 M775.0,153.8 L766.0,153.8  '/>	<g transform="translate(55.3,158.3)" style="stroke:none; fill:black; font-family:Arial; font-size:12.00pt; text-anchor:end">
		<text> 26</text>
	</g>
	<path stroke='black'  d='M63.6,119.6 L72.6,119.6 M775.0,119.6 L766.0,119.6  '/>	<g transform="translate(55.3,124.1)" style="stroke:none; fill:black; font-family:Arial; font-size:12.00pt; text-anchor:end">
		<text> 28</text>
	</g>
	<path stroke='black'  d='M63.6,85.3 L72.6,85.3 M775.0,85.3 L766.0,85.3  '/>	<g transform="translate(55.3,89.8)" style="stroke:none; fill:black; font-family:Arial; font-size:12.00pt; text-anchor:end">
		<text> 30</text>
	</g>
	<path stroke='black'  d='M63.6,51.0 L72.6,51.0 M775.0,51.0 L766.0,51.0  '/>	<g transform="translate(55.3,55.5)" style="stroke:none; fill:black; font-family:Arial; font-size:12.00pt; text-anchor:end">
		<text> 32</text>
	</g>
	<path stroke='black'  d='M63.6,16.7 L72.6,16.7 M775.0,16.7 L766.0,16.7  '/>	<g transform="translate(55.3,21.2)" style="stroke:none; fill:black; font-family:Arial; font-size:12.00pt; text-anchor:end">
		<text> 34</text>
	</g>
	<path stroke='black'  d='M108.1,342.4 L108.1,333.4 M108.1,16.7 L108.1,25.7  '/>	<g transform="translate(108.1,364.9)" style="stroke:none; fill:black; font-family:Arial; font-size:12.00pt; text-anchor:middle">
		<text> 0.1</text>
	</g>
	<path stroke='black'  d='M219.2,342.4 L219.2,333.4 M219.2,16.7 L219.2,25.7  '/>	<g transform="translate(219.2,364.9)" style="stroke:none; fill:black; font-family:Arial; font-size:12.00pt; text-anchor:middle">
		<text> 0.15</text>
	</g>
	<path stroke='black'  d='M330.4,342.4 L330.4,333.4 M330.4,16.7 L330.4,25.7  '/>	<g transform="translate(330.4,364.9)" style="stroke:none; fill:black; font-family:Arial; font-size:12.00pt; text-anchor:middle">
		<text> 0.2</text>
	</g>
	<path stroke='black'  d='M441.5,342.4 L441.5,333.4 M441.5,16.7 L441.5,25.7  '/>	<g transform="translate(441.5,364.9)" style="stroke:none; fill:black; font-family:Arial; font-size:12.00pt; text-anchor:middle">
		<text> 0.25</text>
	</g>
	<path stroke='black'  d='M552.7,342.4 L552.7,333.4 M552.7,16.7 L552.7,25.7  '/>	<g transform="translate(552.7,364.9)" style="stroke:none; fill:black; font-family:Arial; font-size:12.00pt; text-anchor:middle">
		<text> 0.3</text>
	</g>
	<path stroke='black'  d='M663.8,342.4 L663.8,333.4 M663.8,16.7 L663.8,25.7  '/>	<g transform="translate(663.8,364.9)" style="stroke:none; fill:black; font-family:Arial; font-size:12.00pt; text-anchor:middle">
		<text> 0.35</text>
	</g>
	<path stroke='black'  d='M775.0,342.4 L775.0,333.4 M775.0,16.7 L775.0,25.7  '/>	<g transform="translate(775.0,364.9)" style="stroke:none; fill:black; font-family:Arial; font-size:12.00pt; text-anchor:middle">
		<text> 0.4</text>
	</g>
	<path stroke='black'  d='M63.6,16.7 L63.6,342.4 L775.0,342.4 L775.0,16.7 L63.6,16.7 Z  '/>	<g transform="translate(17.6,179.6) rotate(270)" style="stroke:none; fill:black; font-family:Arial; font-size:12.00pt; text-anchor:middle">
		<text>Средняя величина напряжения, $\text{кгс}/\text{мм}^2$</text>
	</g>
	<g transform="translate(419.3,391.9)" style="stroke:none; fill:black; font-family:Arial; font-size:12.00pt; text-anchor:middle">
		<text>Максимальный размер КЭ</text>
	</g>
</g>
	<g id="gnuplot_plot_1" ><title>gnuplot_plot_1</title>
<g style="fill:none; color:#157545; stroke:currentColor; stroke-width:1.00; stroke-linecap:butt; stroke-linejoin:miter">
	<g transform="translate(707.9,39.2)" style="stroke:none; fill:black; font-family:Arial; font-size:12.00pt; text-anchor:end">
		<text>$1.3$</text>
	</g>
	<path stroke-dasharray=' 9,4,1,4,1,4'  d='M716.2,34.7 L758.4,34.7 M775.0,145.6 L552.7,94.9 L330.4,98.9 L219.2,110.3 L174.8,173.9 L130.3,180.3 
		L85.8,222.5  '/>	<use xlink:href='#gpPt4' transform='translate(775.0,145.6) scale(4.50)'/>
	<use xlink:href='#gpPt4' transform='translate(552.7,94.9) scale(4.50)'/>
	<use xlink:href='#gpPt4' transform='translate(330.4,98.9) scale(4.50)'/>
	<use xlink:href='#gpPt4' transform='translate(219.2,110.3) scale(4.50)'/>
	<use xlink:href='#gpPt4' transform='translate(174.8,173.9) scale(4.50)'/>
	<use xlink:href='#gpPt4' transform='translate(130.3,180.3) scale(4.50)'/>
	<use xlink:href='#gpPt4' transform='translate(85.8,222.5) scale(4.50)'/>
	<use xlink:href='#gpPt4' transform='translate(737.3,34.7) scale(4.50)'/>
</g>
	</g>
	<g id="gnuplot_plot_2" ><title>gnuplot_plot_2</title>
<g style="fill:none; color:green; stroke:currentColor; stroke-width:1.00; stroke-linecap:butt; stroke-linejoin:miter">
	<g transform="translate(707.9,57.2)" style="stroke:none; fill:black; font-family:Arial; font-size:12.00pt; text-anchor:end">
		<text>$1.36$</text>
	</g>
	<path stroke-dasharray=' 5,8'  d='M716.2,52.7 L758.4,52.7 M775.0,282.1 L552.7,280.0 L330.4,199.4 L219.2,252.7 L174.8,224.2 L130.3,214.0 
		L85.8,209.8  '/>	<use xlink:href='#gpPt1' transform='translate(775.0,282.1) scale(4.50)'/>
	<use xlink:href='#gpPt1' transform='translate(552.7,280.0) scale(4.50)'/>
	<use xlink:href='#gpPt1' transform='translate(330.4,199.4) scale(4.50)'/>
	<use xlink:href='#gpPt1' transform='translate(219.2,252.7) scale(4.50)'/>
	<use xlink:href='#gpPt1' transform='translate(174.8,224.2) scale(4.50)'/>
	<use xlink:href='#gpPt1' transform='translate(130.3,214.0) scale(4.50)'/>
	<use xlink:href='#gpPt1' transform='translate(85.8,209.8) scale(4.50)'/>
	<use xlink:href='#gpPt1' transform='translate(737.3,52.7) scale(4.50)'/>
</g>
	</g>
	<g id="gnuplot_plot_3" ><title>gnuplot_plot_3</title>
<g style="fill:none; color:blue; stroke:currentColor; stroke-width:1.00; stroke-linecap:butt; stroke-linejoin:miter">
	<g transform="translate(707.9,75.2)" style="stroke:none; fill:black; font-family:Arial; font-size:12.00pt; text-anchor:end">
		<text>$2.33$</text>
	</g>
	<path stroke-dasharray=' 2,4'  d='M716.2,70.7 L758.4,70.7 M775.0,65.9 L552.7,30.8 L330.4,54.1 L219.2,29.6 L174.8,111.9 L130.3,122.5 
		L85.8,121.6  '/>	<use xlink:href='#gpPt2' transform='translate(775.0,65.9) scale(4.50)'/>
	<use xlink:href='#gpPt2' transform='translate(552.7,30.8) scale(4.50)'/>
	<use xlink:href='#gpPt2' transform='translate(330.4,54.1) scale(4.50)'/>
	<use xlink:href='#gpPt2' transform='translate(219.2,29.6) scale(4.50)'/>
	<use xlink:href='#gpPt2' transform='translate(174.8,111.9) scale(4.50)'/>
	<use xlink:href='#gpPt2' transform='translate(130.3,122.5) scale(4.50)'/>
	<use xlink:href='#gpPt2' transform='translate(85.8,121.6) scale(4.50)'/>
	<use xlink:href='#gpPt2' transform='translate(737.3,70.7) scale(4.50)'/>
</g>
	</g>
	<g id="gnuplot_plot_4" ><title>gnuplot_plot_4</title>
<g style="fill:none; color:cyan; stroke:currentColor; stroke-width:1.00; stroke-linecap:butt; stroke-linejoin:miter">
	<g transform="translate(707.9,93.2)" style="stroke:none; fill:black; font-family:Arial; font-size:12.00pt; text-anchor:end">
		<text>$2.34$</text>
	</g>
	<path stroke-dasharray=' 8,4,2,4'  d='M716.2,88.7 L758.4,88.7 M775.0,329.2 L552.7,319.0 L330.4,268.7 L219.2,282.1 L174.8,232.2 L130.3,195.3 
		L85.8,193.5  '/>	<use xlink:href='#gpPt3' transform='translate(775.0,329.2) scale(4.50)'/>
	<use xlink:href='#gpPt3' transform='translate(552.7,319.0) scale(4.50)'/>
	<use xlink:href='#gpPt3' transform='translate(330.4,268.7) scale(4.50)'/>
	<use xlink:href='#gpPt3' transform='translate(219.2,282.1) scale(4.50)'/>
	<use xlink:href='#gpPt3' transform='translate(174.8,232.2) scale(4.50)'/>
	<use xlink:href='#gpPt3' transform='translate(130.3,195.3) scale(4.50)'/>
	<use xlink:href='#gpPt3' transform='translate(85.8,193.5) scale(4.50)'/>
	<use xlink:href='#gpPt3' transform='translate(737.3,88.7) scale(4.50)'/>
</g>
	</g>
<g style="fill:none; color:black; stroke:currentColor; stroke-width:1.00; stroke-linecap:butt; stroke-linejoin:miter">
	<path stroke='black'  d='M63.6,16.7 L63.6,342.4 L775.0,342.4 L775.0,16.7 L63.6,16.7 Z  '/></g>
</g>
</svg>


\caption{Зависимость средних напряжений в отсеках от величины КЭ}
\label{fig:stressToDiscreteness}
\end{figure}

На основании полученных данных и исходя из трудоемкости процесса расчета модели была определена оптимальная величина конечного элемента для дальнейшей работы над моделью, принятая равной $0,11\text{м}$. 
%Ниже приведены картины НДС в месте стыка крыла с фюзеляжем при различных размерах конечного элемента. 