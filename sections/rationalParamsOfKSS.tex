На основе полученных в разделе \ref{sec:creationOfOneModel} данных была проведена оптимизация толщин стенок отсеков и панелей для выполнения требований прочности конструкции. 

\begin{figure}
	\centering
	\def\svgwidth{\textwidth}
	\input{figures/EpureDistributedLoadN1Bottom.pdf_tex}
	\label{fig:epure:loadn1bottom}
	\caption{fig:epure:loadn1bottom}
\end{figure}

\begin{figure}
	\centering
	\def\svgwidth{\textwidth}
	\input{figures/EpureDistributedLoadN1Top.pdf_tex}
	\label{fig:epure:loadn1top}
	\caption{fig:epure:loadn1top}
\end{figure}

\begin{figure}
	\centering
	\def\svgwidth{\textwidth}
	\input{figures/EpureDistributedLoadN2Bottom.pdf_tex}
	\label{fig:epure:loadn2bottom}
	\caption{fig:epure:loadn2bottom}
\end{figure}

\begin{figure}
	\centering
	\def\svgwidth{\textwidth}
	\input{figures/EpureDistributedLoadN2Top.pdf_tex}
	\label{fig:epure:loadn2top}
	\caption{fig:epure:loadn2top}
\end{figure}

\begin{figure}
	\centering
	\def\svgwidth{\textwidth}
	\input{figures/EpureDistributedLoadN12Bottom.pdf_tex}
	\label{fig:epure:loadn12bottom}
	\caption{fig:epure:loadn12bottom}
\end{figure}

\begin{figure}
	\centering
	\def\svgwidth{\textwidth}
	\input{figures/EpureDistributedLoadN12Top.pdf_tex}
	\label{fig:epure:loadn12top}
	\caption{fig:epure:loadn12top}
\end{figure}

\begin{figure}
	\centering
	\def\svgwidth{\textwidth}
	\input{figures/EpureDistributedWeightBottom.pdf_tex}
	\label{fig:epure:loadweightbottom}
	\caption{fig:epure:loadweightbottom}
\end{figure}

\begin{figure}
	\centering
	\def\svgwidth{\textwidth}
	\input{figures/EpureDistributedWeightTop.pdf_tex}
	\label{fig:epure:loadweighttop}
	\caption{fig:epure:loadweighttop}
\end{figure}

%Параметры оптимизированной модели представлены в таблице \ref{tab:paramsOfOptimizedScheme}

%Остальные две главы будут как приложение к этой мощной главе: исследование влияния искривления кессона центроплана на его вес, выбор рациональной КСС.

\tabulinesep = 1mm
\definecolor{lightgray}{gray}{0.9}
\begin{table}[H]
\captionsetup{justification=centering}
\caption{Таблица рациональных параметров}
\begin{tabu}to \linewidth{*2{|X[m c]}|*2{|X[m c]}|}
\hline
Вес фюзеляжа & 618кг & & \\ \hline
Вес крыльев & 1102кг & & \\ \hline
Вес конструкции & 1722кг & & \\ \hline
\end{tabu}
\end{table}
