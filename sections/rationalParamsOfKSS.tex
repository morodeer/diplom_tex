Была выбрана КСС, удовлетворяющая всем компоновочным требованиям, описанным в разделе \ref{Somewhere}. 
На основе полученных в разделе \ref{sec:creationOfOneModel} данных для панелей толщины 1мм, была проведена оптимизация толщин стенок отсеков и панелей до полного удовлетворения требованиям прочности. Параметры оптимизированной модели представлены в таблице \ref{tab:paramsOfOptimizedScheme}

Остальные две главы будут как приложение к этой мощной главе: исследование влияния искривления кессона центроплана на его вес, выбор рациональной КСС.

\tabulinesep = 1mm
\definecolor{lightgray}{gray}{0.9}
\begin{table}[H]
\captionsetup{justification=centering}
\caption{Таблица рациональных параметров}
\begin{tabu}to \linewidth{*2{|X[m c]}|*2{|X[m c]}|}
\hline
Вес фюзеляжа & 618кг & & \\ \hline
Вес крыльев & 1102кг & & \\ \hline
Вес конструкции & 1722кг & & \\ \hline
\end{tabu}
\end{table}

\begin{table}[H]
    %\fontsize{12pt}{14pt}\selectfont
\caption{Зависимость площади панелей центроплана и веса кессона от параметров центроплана относительно варианта с прямым кессоном (данные надо пересчитывать)}
%\rowcolors{2}{}{lightgray}
\begin{tabu}to \linewidth{|c|*4{X[m c]|}*4{X[m c]|}}
\hline
\multirow{2}{*}[-1.1ex]{N} & \multicolumn{4}{c|}{Вес кессона} & \multicolumn{4}{c|}{Площадь панелей центроплана} \\ \cline{2-9}
& Верхние панели & Нижние панели & Боковые стенки & $\Sigma$ & Верхние панели & Нижние панели & Боковые стенки & $\Sigma$ \\
\hline
\taburowcolors {lightgray .. white}
1 & 0.492 & 0.487 & 0.021 & 1.000 & 0.287 & 0.287 & 0.420 & 1.000\\ \hline
2 & 0.373 & 0.393 & 0.046 & 0.811 & 0.287 & 0.288 & 0.547 & 1.126\\ \hline
3 & 0.315 & 0.368 & 0.081 & 0.764 & 0.287 & 0.290 & 0.611 & 1.191\\ \hline
4 & 0.267 & 0.350 & 0.109 & 0.726 & 0.287 & 0.290 & 0.678 & 1.255\\ \hline
5 & 0.243 & 0.331 & 0.111 & 0.684 & 0.287 & 0.292 & 0.745 & 1.322\\ \hline
6 & 0.223 & 0.317 & 0.117 & 0.657 & 0.287 & 0.294 & 0.803 & 1.387\\ \hline
7 & 0.581 & 0.619 & 0.079 & 1.278 & 0.306 & 0.306 & 0.420 & 1.035\\ \hline
8 & 0.420 & 0.429 & 0.088 & 0.937 & 0.306 & 0.299 & 0.547 & 1.154\\ \hline
9 & 0.354 & 0.375 & 0.095 & 0.824 & 0.306 & 0.297 & 0.613 & 1.215\\ \hline
10 & 0.312 & 0.340 & 0.103 & 0.755 & 0.306 & 0.295 & 0.678 & 1.276\\ \hline
11 & 0.289 & 0.325 & 0.110 & 0.723 & 0.306 & 0.292 & 0.745 & 1.341\\ \hline
12 & 0.255 & 0.324 & 0.117 & 0.697 & 0.306 & 0.291 & 0.807 & 1.402\\ \hline
13 & 0.602 & 0.648 & 0.081 & 1.330 & 0.316 & 0.315 & 0.420 & 1.050\\ \hline
14 & 0.427 & 0.456 & 0.088 & 0.971 & 0.316 & 0.308 & 0.549 & 1.172\\ \hline
15 & 0.373 & 0.394 & 0.095 & 0.862 & 0.316 & 0.304 & 0.611 & 1.231\\ \hline
16 & 0.334 & 0.355 & 0.103 & 0.792 & 0.316 & 0.300 & 0.676 & 1.293\\ \hline
17 & 0.284 & 0.337 & 0.110 & 0.731 & 0.316 & 0.298 & 0.741 & 1.355\\ \hline
18 & 0.271 & 0.333 & 0.117 & 0.721 & 0.316 & 0.296 & 0.805 & 1.416\\ \hline
19 & 0.629 & 0.659 & 0.081 & 1.370 & 0.320 & 0.320 & 0.420 & 1.062\\ \hline
20 & 0.442 & 0.463 & 0.088 & 0.993 & 0.320 & 0.313 & 0.547 & 1.181\\ \hline
21 & 0.383 & 0.395 & 0.095 & 0.873 & 0.320 & 0.308 & 0.611 & 1.242\\ \hline
22 & 0.327 & 0.361 & 0.103 & 0.790 & 0.320 & 0.306 & 0.673 & 1.301\\ \hline
23 & 0.317 & 0.341 & 0.110 & 0.768 & 0.320 & 0.301 & 0.743 & 1.363\\ \hline
24 & 0.262 & 0.337 & 0.117 & 0.717 & 0.320 & 0.299 & 0.805 & 1.424\\ \hline
25 & 0.635 & 0.679 & 0.083 & 1.397 & 0.327 & 0.327 & 0.420 & 1.072\\ \hline
26 & 0.462 & 0.477 & 0.088 & 1.027 & 0.327 & 0.318 & 0.547 & 1.192\\ \hline
27 & 0.387 & 0.413 & 0.095 & 0.895 & 0.327 & 0.314 & 0.611 & 1.251\\ \hline
28 & 0.354 & 0.367 & 0.103 & 0.824 & 0.327 & 0.310 & 0.678 & 1.313\\ \hline
29 & 0.299 & 0.348 & 0.110 & 0.756 & 0.327 & 0.307 & 0.743 & 1.373\\ \hline
30 & 0.277 & 0.340 & 0.117 & 0.735 & 0.327 & 0.303 & 0.803 & 1.435\\ \hline
31 & 0.664 & 0.702 & 0.083 & 1.449 & 0.332 & 0.332 & 0.420 & 1.085\\ \hline
32 & 0.472 & 0.492 & 0.089 & 1.053 & 0.332 & 0.322 & 0.549 & 1.205\\ \hline
33 & 0.416 & 0.422 & 0.095 & 0.933 & 0.332 & 0.319 & 0.616 & 1.264\\ \hline
34 & 0.351 & 0.380 & 0.103 & 0.833 & 0.332 & 0.315 & 0.678 & 1.325\\ \hline
35 & 0.316 & 0.357 & 0.110 & 0.784 & 0.332 & 0.312 & 0.743 & 1.383\\ \hline
36 & 0.283 & 0.346 & 0.117 & 0.746 & 0.332 & 0.307 & 0.805 & 1.443\\ \hline
37 & 0.715 & 0.747 & 0.086 & 1.548 & 0.339 & 0.339 & 0.420 & 1.097\\ \hline
38 & 0.482 & 0.507 & 0.090 & 1.078 & 0.339 & 0.329 & 0.547 & 1.214\\ \hline
39 & 0.419 & 0.439 & 0.095 & 0.953 & 0.339 & 0.325 & 0.611 & 1.275\\ \hline
40 & 0.369 & 0.386 & 0.103 & 0.857 & 0.339 & 0.320 & 0.672 & 1.334\\ \hline
41 & 0.326 & 0.361 & 0.110 & 0.797 & 0.339 & 0.317 & 0.738 & 1.394\\ \hline
42 & 0.291 & 0.349 & 0.117 & 0.757 & 0.339 & 0.312 & 0.805 & 1.454\\ \hline

\end{tabu}

\label{tab:KessOptimBigTableNormed}
\end{table}