На основе полученных в разделе \ref{sec:creationOfOneModel} данных была проведена оптимизация толщин стенок отсеков и панелей для выполнения требований прочности конструкции. Эпюра толщин верхних и нижних панелей центроплана представлены на Рис.\ref{fig:epure:weight}.

\begin{figure}[H]
\centering
\captionsetup{justification=centering}
\footnotesize
\begin{subfigure}[b]{0.49\textwidth}
	\centering
	\def\svgwidth{\textwidth}
	
	\input{figures/EpureDistributedLoadN1Top.pdf_tex}
	\label{fig:epure:loadn1top}
	\caption{Эпюра погонных усилий $N_y$ в верхних панелях центроплана}
\end{subfigure}
\begin{subfigure}[b]{0.49\textwidth}
	\centering
	\def\svgwidth{\textwidth}
	\input{figures/EpureDistributedLoadN1Bottom.pdf_tex}
	\label{fig:epure:loadn1bottom}
	\caption{Эпюра погонных усилий $N_y$ в нижних панелях центроплана}
\end{subfigure}
\begin{subfigure}[b]{0.49\textwidth}
	\centering
	\def\svgwidth{\textwidth}
	\input{figures/EpureDistributedWeightTop.pdf_tex}
	\label{fig:epure:loadweighttop}
	\caption{Эпюра погонного веса верхних панелей центроплана}
\end{subfigure}
\begin{subfigure}[b]{0.49\textwidth}
	\centering
	\def\svgwidth{\textwidth}		\input{figures/EpureDistributedWeightBottom.pdf_tex}
	\label{fig:epure:loadweightbottom}
	\caption{Эпюра погонного веса нижних панелей центроплана}
\end{subfigure}
\label{fig:epures}
\end{figure}
%
%
%\begin{figure}[H]
%\centering
%\begin{subfigure}[b]{0.7\textwidth}
%	\centering
%	\def\svgwidth{\textwidth}
%	\input{figures/EpureDistributedLoadN2Top.pdf_tex}
%	\label{fig:epure:loadn2top}
%	\caption{Верхние панели}
%\end{subfigure}
%\begin{subfigure}[b]{0.7\textwidth}
%	\centering
%	\def\svgwidth{\textwidth}	\input{figures/EpureDistributedLoadN2Bottom.pdf_tex}
%	\label{fig:epure:loadn2bottom}
%	\caption{Нижние панели}
%\end{subfigure}
%\label{fig:epure:loadn2}
%\caption{Эпюра $N_y$}
%\end{figure}
%
%\begin{figure}
%	\centering
%	\def\svgwidth{\textwidth}
%	\input{figures/EpureDistributedLoadN12Bottom.pdf_tex}
%	\label{fig:epure:loadn12bottom}
%	\caption{Эпюра $N_\text{xy}$. Нижние панели}
%\end{figure}
%
%\begin{figure}
%	\centering
%	\def\svgwidth{\textwidth}
%	\input{figures/EpureDistributedLoadN12Top.pdf_tex}
%	\label{fig:epure:loadn12top}
%	\caption{Эпюра $N_\text{xy}$. Верхние панели}
%\end{figure}



%Параметры оптимизированной модели представлены в таблице \ref{tab:paramsOfOptimizedScheme}

%Остальные две главы будут как приложение к этой мощной главе: исследование влияния искривления кессона центроплана на его вес, выбор рациональной КСС.

\tabulinesep = 1mm
\definecolor{lightgray}{gray}{0.9}
\begin{table}[H]
\captionsetup{justification=centering}
\caption{Таблица рациональных параметров}
\begin{tabu}to \linewidth{*2{|X[m c]}|}
\hline
Вес фюзеляжа & 618кг \\ \hline
Вес крыльев & 1102кг \\ \hline
Вес конструкции & 1722кг \\ \hline
\end{tabu}
\end{table}
