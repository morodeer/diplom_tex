В работе был рассмотрен вопрос проектировки беспилотного летательного аппарата (БПЛА), предназаченного для длительного ($\approx24$~часа без дозаправки) барражирования на высоте $\approx22$~км в целях мониторинга и разведки. В связи с этим к БПЛА были предъявлены высокие требования по малозаметности и аэродинамическим свойствам. Конструкторами была представлена модель БПЛА, отвечающая требованиям аэродинамики и малозаметности. Для этой модели расчетный вес конструкции планера -- 3520~кг. Вид фюзеляжа представлен на Рис.\ref{fig:BPLA_TSAGI}. 

\begin{figure}[H]
\centering
\includegraphics[width=0.6\textwidth]{BPS_Catia}
\caption{} фюзеляжа БПЛА-ЦАГИ}
\label{fig:BPLA_TSAGI}
\end{figure}

Была поставлена задача построения расчетной модели данного БПЛА и дальнейшего прочностного анализа модели. При построении модели необходимо было учесть дальнейшее возможное изменение формы обводов и, как следствие, изменение аэродинамических нагрузок.
 

%Не забыть про то, что мы также хотим менять аэродинамику
%Требования: БПЛА, полет на таких-то высотах, столько-то. Весовая сводка такая-то, максимальные перегрузки, коэффициент запаса, аэродинамика. Ограничения - малозаметность, вес, пожаробезопасность отсека двигателя. 
%

