\chapter{Проблема компоновки самолета по схеме ``летающее крыло'' и её решение}

В настоящее время всё большее внимание уделяется принципиальной схеме самолета ``летающее крыло''. Данная схема применяется в том числе и для разработки беспилотных летательных аппаратов, предназначенных для разведки. В конструктировании таких самолетов особое внимание уделяется требованиям малозаметности и увеличения аэродинамического качества, и как следствие, возможности барражировать в течение длительного времени. 

Для удовлетворения данным требованиям конструкцию самолета создают максимально ``плоской'' -- так, в подобных конструкциях строительная высота фюзеляжа сравнима с высотой двигателя. Один из способов создания подобной конструкции -- использование изогнутого кессона. (Рис.\ref{fig:OriginalSectionWithEngine}). Примером такого самолета служит концепт американского беспилотного летательного аппарата RQ-180 (Рис.\ref{fig:rq180}). 


\begin{figure}[ht]
\centering
\includegraphics[width=0.6\textwidth]{rq180concept}
\caption{Концепт американского БПЛА RQ-180 \cite{AvWeekUAV}}
\label{fig:rq180}
\end{figure}


 \section{Оценка возможных потерь, вносимых изогнутой формой кессона}



