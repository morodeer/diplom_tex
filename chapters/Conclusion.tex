\chapter*{Выводы}

Сформированы основные базовые требования к проведению многодисциплинарного проектирования перспективной гипотетической конструкции БПЛА с крылом большого удлинения и криволинейной формой центроплана. 

Была обоснована необходимость:
\begin{itemize}
\item использования параметрической МКЭ-модели большой размерности всей конструкции БПЛА;
\item решения модельной задачи по определению зависимости веса конструкции БПЛА от геометрических параметров, определяющих форму искривленного центроплана;
\item выбора рациональной КСС корневой части кабины БПЛА в зоне крепления двигателя.
\end{itemize}

Построена параметрическая МКЭ-модель большой размерности гипотетической конструкции БПЛА для проведения проектировочных исследований по поиску рациональных проектных параметров конструкции, обеспечивающих минимальные весовые характеристики гипотетической конструкции БПЛА. 

Модель позволяет проводить исследования прочности конструкции гипотетического БПЛА как для металлических, так и для композиционных конструкционных материалов (в бакалаврской работе были рассмотрены только металлические варианты конструкции). При построении конечноэлементной модели исползовались коммерческие программные комплексы: patran, nastran, а также программные комплексы, разработанные в ЦАГИ: конвер, (Крючков) и (Фомин).
Проектировочная модель  включала свыше 30 базовых варьируемых параметров. 

Найдена рациональная размерность конечноэлементной модели с характерным размером конечного элемента, равным 0.11м, и количеством конечных элементов порядка $200000$. 

Валидационные исследования, проведенные в рамках параметрической МКЭ-модели, показали ее высокую точность при определении локальных параметров НДС, а также хорошее соответствие результатов с результатами, полученных на альтернативных моделях и МКЭ-модели конструкции БПЛА-ЦАГИ. 


Решена модельная задача по определению зависимости веса конструкции искривленного центроплана от базовых геометрических параметров, определяющих форму центроплана: максимальной строительной высоты центроплана и параметра, характеризующего кривизну центроплана (расстояние от средней горизонтали ЛА до нижней точки сечения). Получены рациональные значения базовых параметров, реализующие минимум веса конструкции центроплана. Соответствующие параметры равны: . Представлены результаты (весовые характеристики конструкции центроплана) для 42 комбинаций данных параметров, которые могут быть использованы в дальнейшем для решения многодисциплинарной проектировочной задачи с изменением геометрических параметров, формирующих внешние обводы. 

Аналогичное: проведены сравнительные весовые исследования трех альтернативных КСС гипотетической конструкции БПЛА с различными схемами организации крепления двигателя. Описать целиком. 