\chapter*{Заключение}

В работе сформированы основные базовые требования к проведению многодисциплинарного проектирования перспективной гипотетической конструкции БПЛА с крылом большого удлинения и криволинейной формой центроплана. 

Была обоснована необходимость:
\begin{itemize}
\item использования параметрической МКЭ-модели большой размерности всей конструкции БПЛА;
\item решения модельной задачи по определению зависимости веса конструкции БПЛА от геометрических параметров, определяющих форму искривленного центроплана;
\item выбора рациональной КСС корневой части кабины БПЛА в зоне крепления двигателя.
\end{itemize}

Построена параметрическая МКЭ-модель большой размерности гипотетической конструкции БПЛА для проведения проектировочных исследований по поиску рациональных проектных параметров конструкции, обеспечивающих минимальные весовые характеристики гипотетической конструкции БПЛА при удовлетворении условий по прочности. 

Созданная модель позволяет проводить исследования прочности конструкции гипотетического БПЛА как для металлических, так и для композиционных конструкционных материалов (в бакалаврской работе были рассмотрены только металлические варианты конструкции). При построении конечно-элементной модели использовались коммерческие программные комплексы: patran, nastran, а также программные комплексы, разработанные в ЦАГИ: конвер, (Крючков).
Проектировочная модель  включала свыше 30 базовых варьируемых параметров. 

Найдена рациональная размерность конечно-элементной модели с характерным размером конечного элемента, равным 0.11м, и количеством конечных элементов около $200000$. 

Валидационные исследования, проведенные в рамках параметрической МКЭ-модели, подтвердили ее высокую точность при определении параметров локального НДС, а также хорошее соответствие результатов расчетов с использованием этой параметрической модели с результатами, полученными на альтернативных моделях и МКЭ-модели конструкции БПЛА-ЦАГИ. 


Решена модельная задача по определению зависимости веса конструкции искривленного центроплана от базовых геометрических параметров, определяющих форму центроплана: максимальной строительной высоты центроплана и параметра, характеризующего кривизну центроплана (расстояние от средней горизонтали ЛА до нижней точки сечения). Получены рациональные значения этих базовых параметров, реализующие минимум веса конструкции центроплана. Соответствующие параметры равны: $y_\text{отн} = 0\text{м}$, $h_\text{стр} = 1.4\text{м}$. Представлены результаты (весовые характеристики конструкции центроплана) для 42 комбинаций исследуемых базовых параметров, которые могут быть использованы в дальнейшем для решения многодисциплинарной проектировочной задачи с включением в варьируемые проектировочные параметры величин, характеризующих изменение геометрических параметров, формирующих внешние обводы. 

Проведены сравнительные весовые исследования трех альтернативных КСС гипотетической конструкции с различными силовыми элементами конструкции, обеспечивающих крепление к центроплану хвостовой части. Была определена величина максимального размера КЭ, используемого в модели, обеспечивающая рациональное соотношение точности расчета и трудоемкости процесса расчета, равная $0.11\m$. Были сформированы три МКЭ-модели для альтернативных КСС гипотетической конструкции. На основе сформированных моделей была проведена оптимизация толщин панелей для удовлетворения условиям прочности для каждой из моделей. Получены сравнительные весовые характеристики альтернативных КСС гипотетической конструкции. Найден оптимальный из предложенных вариант КСС (Вариант 3 в разделе \ref{sec:modelsComparison}, Рис.\ref{fig:variants_mke:3}). 

Решения, полученные в данной работе для конструкции гипотетического БПЛА, были использованы в практической работе по проектированию конструкции БПЛА-ЦАГИ.